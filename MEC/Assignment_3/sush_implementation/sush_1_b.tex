% This LaTeX was auto-generated from MATLAB code.
% To make changes, update the MATLAB code and export to LaTeX again.

\documentclass{article}

\usepackage[utf8]{inputenc}
\usepackage[T1]{fontenc}
\usepackage{lmodern}
\usepackage{graphicx}
\usepackage{color}
\usepackage{hyperref}
\usepackage{amsmath}
\usepackage{amsfonts}
\usepackage{epstopdf}
\usepackage[table]{xcolor}
\usepackage{matlab}

\sloppy
\epstopdfsetup{outdir=./}
\graphicspath{ {./sush_1_b_images/} }

\begin{document}

\matlabtitle{Problem Set 3: Manipulation Estimation and Control}

\matlabheading{1. Dead Reckoning and Kalman Filter}


\vspace{1em}
\matlabheadingtwo{1. a. linearized approximation of the system as a function of the current state}


\vspace{1em}
\begin{par}
\begin{flushleft}
\includegraphics[width=\maxwidth{68.13848469643753em}]{image_0}
\end{flushleft}
\end{par}


\vspace{1em}
\begin{par}
\begin{flushleft}
\includegraphics[width=\maxwidth{68.03813346713497em}]{image_1}
\end{flushleft}
\end{par}

\begin{par}
\begin{flushleft}
\includegraphics[width=\maxwidth{69.24234821876568em}]{image_2}
\end{flushleft}
\end{par}

\begin{par}
\begin{flushleft}
\includegraphics[width=\maxwidth{69.34269944806825em}]{image_3}
\end{flushleft}
\end{par}


\vspace{1em}

\vspace{1em}
\matlabheadingtwo{1. b. Estimation of Covariance Matrix for process and measurement noise}

\begin{matlabcode}
% import the calibrate.mat file to get:
% Train time params
% - t_groundtruth (absolute time values)
% - q_groundtruth (robot states at the above mentioned time values)
% - u (inputs to robot at time values without any noise)
% 
% Test time values
% - t_y (a vector of times associated with the GPS measurement)
% - y (noisy GPS measurements taken at the times in t y)

load CMU/Assignment_Sem_1/MEC/Assignment_3/'Problem1 (EKF)'/calibration.mat
\end{matlabcode}


\vspace{1em}
\begin{par}
\begin{flushleft}
The error between noisy measurement from actual sensor and groundtruth is recorded. This is then used to calculate the covariance of the measurement noise. 
\end{flushleft}
\end{par}

\begin{matlabcode}
% get length of time vector
time_duration = size(t_groundTruth,2);

% create a dummy array which will hold the error values
measure_errors = zeros(2,250);

for t = 1:time_duration
    if mod(t,10) == 0
        time = t/10;
        measure_errors(:,time) = (y(:,time) - q_groundTruth(1:2,t));
    end
end

W = cov(measure_errors.')
\end{matlabcode}
\begin{matlaboutput}
W = 2x2    
    1.8817    0.0632
    0.0632    2.1384

\end{matlaboutput}


\vspace{1em}
\begin{par}
\begin{flushleft}
To calculate the process noise, we use the linearized system model derived in \textbf{1.a.}
\end{flushleft}
\end{par}

\begin{par}
\begin{flushleft}
\textbf{\textless{}INSERT 1.a. equation\textgreater{}}
\end{flushleft}
\end{par}

\begin{matlabcode}
% create a dummy array which will hold the error values
process_errors = zeros(2,2500);
timestep = 0.01
\end{matlabcode}
\begin{matlaboutput}
timestep = 0.0100
\end{matlaboutput}
\begin{matlabcode}

for t = 1:time_duration-1
    % time = t/10;
    
    % calculate the F matrix:
    F = [1 0 -timestep*u(1,t)*sin(q_groundTruth(3,t));
         0 1 timestep*u(1,t)*cos(q_groundTruth(3,t));
         0 0 1];
    
    % calculate the G matrix:
    G = [timestep*cos(q_groundTruth(3,t))  0;
         timestep*sin(q_groundTruth(3,t))   0;
         0                          timestep];

    gamma = G;

    process_errors(:,t) = gamma \ (q_groundTruth(:,t+1) - F*q_groundTruth(:,t) - G*u(:,t));
end

V = cov(process_errors.')
\end{matlabcode}
\begin{matlaboutput}
V = 2x2    
    0.2591    0.0010
    0.0010    0.0625

\end{matlaboutput}

\end{document}
