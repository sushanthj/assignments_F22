% This LaTeX was auto-generated from MATLAB code.
% To make changes, update the MATLAB code and export to LaTeX again.

\documentclass{article}

\usepackage[utf8]{inputenc}
\usepackage[T1]{fontenc}
\usepackage{lmodern}
\usepackage{graphicx}
\usepackage{color}
\usepackage{hyperref}
\usepackage{amsmath}
\usepackage{amsfonts}
\usepackage{epstopdf}
\usepackage[table]{xcolor}
\usepackage{matlab}

\sloppy
\epstopdfsetup{outdir=./}
\graphicspath{ {./q6_images/} }

\begin{document}

\matlabheading{Q2. Fun with SE(2)}

\vspace{1em}

\matlabheadingthree{(a.) A, relative to fixed frame}


\vspace{1em}
\matlabheadingthree{(b.) A, relative to the fixed frame, followed by B, relative to the current frame.}


\vspace{1em}
\matlabheadingthree{(c.) A, relative to the fixed frame, followed by B, relative to the fixed frame.}


\vspace{1em}
\matlabheadingthree{(d.) B, relative to the fixed frame.}


\vspace{1em}
\matlabheadingthree{(e.) B, relative to the fixed frame, followed by A, relative to the fixed frame.}


\vspace{1em}
\matlabheadingthree{(f.) B, relative to the fixed frame, followed by A, relative to the current frame.}



\vspace{1em}
\vspace{1em}

\matlabheading{Q6. Jacobian of Given Manipulator}


\vspace{1em}
\matlabheadingtwo{1. Direct Differentiation Method}

\begin{matlabcode}
% Creating symbolic variables for the manipulator
syms theta1 d2 theta3

% The DH matrix
DH = [[theta1, 10+d2, 0,  0];
      [0,      0,     9,  pi/2];
      [theta3, 0,     5,  0]]
\end{matlabcode}
\begin{matlabsymbolicoutput}
DH = 

\hskip1em $\displaystyle \left(\begin{array}{cccc}
\theta_1  & d_2 +10 & 0 & 0\\
0 & 0 & 9 & \frac{\pi }{2}\\
\theta_3  & 0 & 5 & 0
\end{array}\right)$
\end{matlabsymbolicoutput}
\begin{matlabcode}

num_joints = size(DH,1)
\end{matlabcode}
\begin{matlaboutput}
num_joints = 3
\end{matlaboutput}
\begin{matlabcode}

for joints = 1:num_joints
    
    t_i = DH(joints,1); % theta_i
    d_i = DH(joints,2);
    a_i = DH(joints,3);
    alpha_i = DH(joints,4);

    H_mat = [ 
              [cos(t_i), -sin(t_i)*cos(alpha_i), sin(t_i)*sin(alpha_i),  a_i*cos(t_i)];
              [sin(t_i), cos(t_i)*cos(alpha_i),  -cos(t_i)*sin(alpha_i), a_i*sin(t_i)];
              [0,        sin(alpha_i),           cos(alpha_i),           d_i         ];
              [0,        0,                      0,                      1           ] 
            ];
    if joints == 1
        H_1 = H_mat
    elseif joints == 2
        H_2 = H_mat
    else
        H_3 = H_mat
    end
end
\end{matlabcode}
\begin{matlabsymbolicoutput}
H\_1 = 

\hskip1em $\displaystyle \left(\begin{array}{cccc}
\cos \left(\theta_1 \right) & -\sin \left(\theta_1 \right) & 0 & 0\\
\sin \left(\theta_1 \right) & \cos \left(\theta_1 \right) & 0 & 0\\
0 & 0 & 1 & d_2 +10\\
0 & 0 & 0 & 1
\end{array}\right)$
H\_2 = 

\hskip1em $\displaystyle \left(\begin{array}{cccc}
1 & 0 & 0 & 9\\
0 & 0 & -1 & 0\\
0 & 1 & 0 & 0\\
0 & 0 & 0 & 1
\end{array}\right)$
H\_3 = 

\hskip1em $\displaystyle \left(\begin{array}{cccc}
\cos \left(\theta_3 \right) & -\sin \left(\theta_3 \right) & 0 & 5\,\cos \left(\theta_3 \right)\\
\sin \left(\theta_3 \right) & \cos \left(\theta_3 \right) & 0 & 5\,\sin \left(\theta_3 \right)\\
0 & 0 & 1 & 0\\
0 & 0 & 0 & 1
\end{array}\right)$
\end{matlabsymbolicoutput}
\begin{matlabcode}

% Finding the final transformation matrix

H_end_to_base = H_1*H_2*H_3
\end{matlabcode}
\begin{matlabsymbolicoutput}
H\_end\_to\_base = 

\hskip1em $\displaystyle \left(\begin{array}{cccc}
\cos \left(\theta_1 \right)\,\cos \left(\theta_3 \right) & -\cos \left(\theta_1 \right)\,\sin \left(\theta_3 \right) & \sin \left(\theta_1 \right) & 9\,\cos \left(\theta_1 \right)+5\,\cos \left(\theta_1 \right)\,\cos \left(\theta_3 \right)\\
\cos \left(\theta_3 \right)\,\sin \left(\theta_1 \right) & -\sin \left(\theta_1 \right)\,\sin \left(\theta_3 \right) & -\cos \left(\theta_1 \right) & 9\,\sin \left(\theta_1 \right)+5\,\cos \left(\theta_3 \right)\,\sin \left(\theta_1 \right)\\
\sin \left(\theta_3 \right) & \cos \left(\theta_3 \right) & 0 & d_2 +5\,\sin \left(\theta_3 \right)+10\\
0 & 0 & 0 & 1
\end{array}\right)$
\end{matlabsymbolicoutput}
\begin{matlabcode}

f_theta = H_end_to_base(1:3,4)
\end{matlabcode}
\begin{matlabsymbolicoutput}
f\_theta = 

\hskip1em $\displaystyle \left(\begin{array}{c}
9\,\cos \left(\theta_1 \right)+5\,\cos \left(\theta_1 \right)\,\cos \left(\theta_3 \right)\\
9\,\sin \left(\theta_1 \right)+5\,\cos \left(\theta_3 \right)\,\sin \left(\theta_1 \right)\\
d_2 +5\,\sin \left(\theta_3 \right)+10
\end{array}\right)$
\end{matlabsymbolicoutput}
\begin{matlabcode}

jacob = jacobian(f_theta, [theta1, d2, theta3])
\end{matlabcode}
\begin{matlabsymbolicoutput}
jacob = 

\hskip1em $\displaystyle \left(\begin{array}{ccc}
-9\,\sin \left(\theta_1 \right)-5\,\cos \left(\theta_3 \right)\,\sin \left(\theta_1 \right) & 0 & -5\,\cos \left(\theta_1 \right)\,\sin \left(\theta_3 \right)\\
9\,\cos \left(\theta_1 \right)+5\,\cos \left(\theta_1 \right)\,\cos \left(\theta_3 \right) & 0 & -5\,\sin \left(\theta_1 \right)\,\sin \left(\theta_3 \right)\\
0 & 1 & 5\,\cos \left(\theta_3 \right)
\end{array}\right)$
\end{matlabsymbolicoutput}



\vspace{1em}
\matlabheadingtwo{2. Column-by-column Method}

\matlabheadingthree{Building the 1st column of the Jacobian (revolute)}

\begin{par}
\hfill \break
\end{par}

\begin{matlabcode}
% Find R
R_0_to_0 = eye(3);

d_3_to_0 = H_end_to_base(1:3,4);

v_1_to_0 = R_0_to_0 * cross([0;0;1], d_3_to_0);
\end{matlabcode}

\vspace{1em}

\matlabheadingthree{Building the 2nd column of the Jacobian (prismatic)}

\begin{par}
\hfill \break
\end{par}

\begin{matlabcode}
% Find R
R_1_to_0 = H_1(1:3,1:3); % H_1 was found previously

v_2_to_0 = R_1_to_0 * [0;0;1];
\end{matlabcode}


\vspace{1em}
\matlabheadingthree{Building the 3rd column of the Jacobian (prismatic)}

\begin{par}
\hfill \break
\end{par}

\begin{matlabcode}
% Find R
H_2_to_0 = H_1 * H_2;
R_2_to_0 = H_2_to_0(1:3,1:3);

d_3_to_2 = H_3(1:3,4);

v_3_to_0 = R_2_to_0 * cross([0;0;1], d_3_to_2);
\end{matlabcode}


\vspace{1em}
\matlabheadingthree{Combining the Columns}

\begin{par}
\hfill \break
\end{par}

\begin{matlabcode}
% We can see that the second jacobian (column-by-column building method)
% will be the same as the first jacobian (direct differentiation method)
jacob_2 = [v_1_to_0, v_2_to_0, v_3_to_0]
\end{matlabcode}
\begin{matlabsymbolicoutput}
jacob\_2 = 

\hskip1em $\displaystyle \left(\begin{array}{ccc}
-9\,\sin \left(\theta_1 \right)-5\,\cos \left(\theta_3 \right)\,\sin \left(\theta_1 \right) & 0 & -5\,\cos \left(\theta_1 \right)\,\sin \left(\theta_3 \right)\\
9\,\cos \left(\theta_1 \right)+5\,\cos \left(\theta_1 \right)\,\cos \left(\theta_3 \right) & 0 & -5\,\sin \left(\theta_1 \right)\,\sin \left(\theta_3 \right)\\
0 & 1 & 5\,\cos \left(\theta_3 \right)
\end{array}\right)$
\end{matlabsymbolicoutput}
\begin{matlabcode}

% substituting actual values of theta1 and theta3 into jacobian we get:
theta1 = 0;
theta3 = 0;

% subs(jacob_2)
\end{matlabcode}


\matlabheadingtwo{3. Singular Configurations}

\begin{par}
\begin{flushleft}
Singular Configurations can be found by checking for a rank reduction in the Jacobian. This is done by finding
\end{flushleft}
\end{par}

\begin{par}
\begin{flushleft}
the determinant of the Jacobian and setting its value to zero:
\end{flushleft}
\end{par}

\begin{matlabcode}
jacob_det = det(jacob);

eqn = jacob_det == 0
\end{matlabcode}
\begin{matlabsymbolicoutput}
eqn = 

\hskip1em $\displaystyle -45\,{\cos \left(\theta_1 \right)}^2 \,\sin \left(\theta_3 \right)-45\,{\sin \left(\theta_1 \right)}^2 \,\sin \left(\theta_3 \right)-25\,{\cos \left(\theta_1 \right)}^2 \,\cos \left(\theta_3 \right)\,\sin \left(\theta_3 \right)-25\,\cos \left(\theta_3 \right)\,{\sin \left(\theta_1 \right)}^2 \,\sin \left(\theta_3 \right)=0$
\end{matlabsymbolicoutput}

\end{document}
